%!TEX TS-program = xelatex

% Шаблон документа LaTeX создан в 2018 году
% Алексеем Подчезерцевым
% В качестве исходных использованы шаблоны
% 	Данилом Фёдоровых (danil@fedorovykh.ru) 
%		https://www.writelatex.com/coursera/latex/5.2.2
%	LaTeX-шаблон для русской кандидатской диссертации и её автореферата.
%		https://github.com/AndreyAkinshin/Russian-Phd-LaTeX-Dissertation-Template

\documentclass[a4paper,14pt]{article}


%%% Работа с русским языком
\usepackage[english,russian]{babel}   %% загружает пакет многоязыковой вёрстки
\usepackage{fontspec}      %% подготавливает загрузку шрифтов Open Type, True Type и др.
\defaultfontfeatures{Ligatures={TeX},Renderer=Basic}  %% свойства шрифтов по умолчанию
\setmainfont[Ligatures={TeX,Historic}]{Times New Roman} %% задаёт основной шрифт документа
\setsansfont{Comic Sans MS}                    %% задаёт шрифт без засечек
\setmonofont{Courier New}
\usepackage{indentfirst}
\frenchspacing

\renewcommand{\epsilon}{\ensuremath{\varepsilon}}
\renewcommand{\phi}{\ensuremath{\varphi}}
\renewcommand{\kappa}{\ensuremath{\varkappa}}
\renewcommand{\le}{\ensuremath{\leqslant}}
\renewcommand{\leq}{\ensuremath{\leqslant}}
\renewcommand{\ge}{\ensuremath{\geqslant}}
\renewcommand{\geq}{\ensuremath{\geqslant}}
\renewcommand{\emptyset}{\varnothing}

%%% Дополнительная работа с математикой
\usepackage{amsmath,amsfonts,amssymb,amsthm,mathtools} % AMS
\usepackage{icomma} % "Умная" запятая: $0,2$ --- число, $0, 2$ --- перечисление

%% Номера формул
%\mathtoolsset{showonlyrefs=true} % Показывать номера только у тех формул, на которые есть \eqref{} в тексте.
%\usepackage{leqno} % Нумерация формул слева	

%% Перенос знаков в формулах (по Львовскому)
\newcommand*{\hm}[1]{#1\nobreak\discretionary{}
	{\hbox{$\mathsurround=0pt #1$}}{}}

%%% Работа с картинками
\usepackage{graphicx}  % Для вставки рисунков
\graphicspath{{images/}}  % папки с картинками
\setlength\fboxsep{3pt} % Отступ рамки \fbox{} от рисунка
\setlength\fboxrule{1pt} % Толщина линий рамки \fbox{}
\usepackage{wrapfig} % Обтекание рисунков текстом

%%% Работа с таблицами
\usepackage{array,tabularx,tabulary,booktabs} % Дополнительная работа с таблицами
\usepackage{longtable}  % Длинные таблицы
\usepackage{multirow} % Слияние строк в таблице
\usepackage{float}% http://ctan.org/pkg/float

%%% Программирование
\usepackage{etoolbox} % логические операторы


%%% Страница
\usepackage{extsizes} % Возможность сделать 14-й шрифт
\usepackage{geometry} % Простой способ задавать поля
\geometry{top=20mm}
\geometry{bottom=20mm}
\geometry{left=20mm}
\geometry{right=10mm}
%
%\usepackage{fancyhdr} % Колонтитулы
% 	\pagestyle{fancy}
%\renewcommand{\headrulewidth}{0pt}  % Толщина линейки, отчеркивающей верхний колонтитул
% 	\lfoot{Нижний левый}
% 	\rfoot{Нижний правый}
% 	\rhead{Верхний правый}
% 	\chead{Верхний в центре}
% 	\lhead{Верхний левый}
%	\cfoot{Нижний в центре} % По умолчанию здесь номер страницы

\usepackage{setspace} % Интерлиньяж
\onehalfspacing % Интерлиньяж 1.5
%\doublespacing % Интерлиньяж 2
%\singlespacing % Интерлиньяж 1

\usepackage{lastpage} % Узнать, сколько всего страниц в документе.

\usepackage{soul} % Модификаторы начертания

\usepackage{hyperref}
\usepackage[usenames,dvipsnames,svgnames,table,rgb]{xcolor}
\hypersetup{				% Гиперссылки
	unicode=true,           % русские буквы в раздела PDF
	pdftitle={Проект по БД},   % Заголовок
	pdfauthor={Подчезерцев Алексей, Солодянкин Андрей},      % Автор
	pdfsubject={ДЗ по БД},      % Тема
	pdfcreator={Подчезерцев Алексей, Солодянкин Андрей}, % Создатель
	pdfproducer={Подчезерцев Алексей, Солодянкин Андрей}, % Производитель
	pdfkeywords={БД} {SQL} {PostgreSQL}, % Ключевые слова
	colorlinks=true,       	% false: ссылки в рамках; true: цветные ссылки
	linkcolor=black,          % внутренние ссылки
	citecolor=black,        % на библиографию
	filecolor=magenta,      % на файлы
	urlcolor=black           % на URL
}
\makeatletter 
\def\@biblabel#1{#1. } 
\makeatother
\usepackage{cite} % Работа с библиографией
%\usepackage[superscript]{cite} % Ссылки в верхних индексах
%\usepackage[nocompress]{cite} % 
\usepackage{csquotes} % Еще инструменты для ссылок

\usepackage{multicol} % Несколько колонок

\usepackage{tikz} % Работа с графикой
\usepackage{pgfplots}
\usepackage{pgfplotstable}

% ГОСТ заголовки
\usepackage[font=small]{caption}
%\captionsetup[table]{justification=centering, labelsep = newline} % Таблицы по правобу краю
%\captionsetup[figure]{justification=centering} % Картинки по центру


\newcommand{\tablecaption}[1]{\addtocounter{table}{1}\small \begin{flushright}\tablename \ \thetable\end{flushright}%	
\begin{center}#1\end{center}}

\newcommand{\imref}[1]{Рис.~\ref{#1}}

\usepackage{multirow}
\usepackage{spreadtab}
\newcolumntype{K}[1]{@{}>{\centering\arraybackslash}p{#1cm}@{}}


\usepackage{xparse}
\ExplSyntaxOn
\DeclareExpandableDocumentCommand{\juliandate}{ m m m }
{
	\juliandate_calc:nnnn { #1 } { #2 } { #3 } { \use:n }
}
\NewDocumentCommand{\storejuliandate}{ s m m m m }
{
	\IfBooleanTF{#1}
	{
		\juliandate_calc:nnnn { #3 } { #4 } { #5 } { \cs_set:Npx #2 }
	}
	{
		\juliandate_calc:nnnn { #3 } { #4 } { #5 } { \cs_new:Npx #2 }
	}
}
\cs_new:Npn \juliandate_calc:nnnn #1 #2 #3 #4 % #1 = day, #2 = month, #3 = year, #4 = what to do
{
	#4 
	{
		\int_eval:n
		{
			#1 +
			\int_div_truncate:nn { 153 * (#2 + 12 * \int_div_truncate:nn { 14 - #2 } { 12 } - 3) + 2 } { 5 } +
			365 * (#3 + 4800 - \int_div_truncate:nn { 14 - #2 } { 12 } ) +
			\int_div_truncate:nn { #3 + 4800 - \int_div_truncate:nn { 14 - #2 } { 12 } } { 4 } -
			\int_div_truncate:nn { #3 + 4800 - \int_div_truncate:nn { 14 - #2 } { 12 } } { 100 } + 
			\int_div_truncate:nn { #3 + 4800 - \int_div_truncate:nn { 14 - #2 } { 12 } } { 400 } -
			32045
		}
	}
}

\tl_new:N \l__juliandate_g_tl
\tl_new:N \l__juliandate_dg_tl
\tl_new:N \l__juliandate_c_tl
\tl_new:N \l__juliandate_dc_tl
\tl_new:N \l__juliandate_b_tl
\tl_new:N \l__juliandate_db_tl
\tl_new:N \l__juliandate_a_tl
\tl_new:N \l__juliandate_da_tl
\tl_new:N \l__juliandate_y_tl
\tl_new:N \l__juliandate_m_tl
\tl_new:N \l__juliandate_d_tl
\int_new:N \l_juliandate_day_int
\int_new:N \l_juliandate_month_int
\int_new:N \l_juliandate_year_int

\cs_new:Npn \__juliandate_set:nn #1 #2
{
	\tl_set:cx { l__juliandate_#1_tl } { \int_eval:n { #2 } }
}
\cs_new:Npn \__juliandate_use:n #1
{
	\tl_use:c { l__juliandate_#1_tl }
}
\cs_new_protected:Npn \juliandate_reverse:n #1
{
	\__juliandate_set:nn { g }
	{ \int_div_truncate:nn { #1 + 32044 } { 146097 } }
	\__juliandate_set:nn { dg }
	{ \int_mod:nn { #1 + 32044 } { 146097 } }
	\__juliandate_set:nn { c }
	{ \int_div_truncate:nn { ( \int_div_truncate:nn { \__juliandate_use:n { dg } } { 36524 } + 1) * 3 } { 4 } }
	\__juliandate_set:nn { dc }
	{ \__juliandate_use:n { dg } - \__juliandate_use:n { c } * 36524 }
	\__juliandate_set:nn { b }
	{ \int_div_truncate:nn { \__juliandate_use:n { dc } } { 1461 } }
	\__juliandate_set:nn { db }
	{ \int_mod:nn { \__juliandate_use:n { dc } } { 1461 } }
	\__juliandate_set:nn { a }
	{ \int_div_truncate:nn { ( \int_div_truncate:nn { \__juliandate_use:n { db } } { 365 } + 1) * 3 } { 4 } }
	\__juliandate_set:nn { da }
	{ \__juliandate_use:n { db } - \__juliandate_use:n { a } * 365 }
	\__juliandate_set:nn { y }
	{
		\__juliandate_use:n { g } * 400 + 
		\__juliandate_use:n { c } * 100 + 
		\__juliandate_use:n { b } * 4 + 
		\__juliandate_use:n { a }
	}
	\__juliandate_set:nn { m }
	{ \int_div_truncate:nn { \__juliandate_use:n { da } * 5 + 308 } { 153 } - 2 }
	\__juliandate_set:nn { d }
	{ \__juliandate_use:n { da } - \int_div_truncate:nn { (\__juliandate_use:n { m } + 4) * 153 } { 5 } + 122 }
	\int_set:Nn \l_juliandate_year_int
	{ \__juliandate_use:n { y } - 4800 + \int_div_truncate:nn { \__juliandate_use:n { m } + 2 } { 12 } }
	\int_set:Nn \l_juliandate_month_int
	{ \int_mod:nn { \__juliandate_use:n { m } + 2 } { 12 } + 1 }
	\int_set:Nn \l_juliandate_day_int
	{ \__juliandate_use:n { d } + 1 }
}
\cs_generate_variant:Nn \juliandate_reverse:n { x }

\NewDocumentCommand{\showday}{ m }
{
	\juliandate_reverse:n { #1 }
	\int_to_arabic:n { \l_juliandate_day_int }-
	\int_to_arabic:n { \l_juliandate_month_int }-
	\int_to_arabic:n { \l_juliandate_year_int }
}

\NewDocumentCommand{\tomorrow}{ }
{
	\group_begin:
	\juliandate_reverse:x { \juliandate_calc:nnnn { \day + 1 } { \month } { \year } { \use:n } }
	\day = \l_juliandate_day_int
	\month = \l_juliandate_month_int
	\year = \l_juliandate_year_int
	\today
	\group_end:
}
\NewDocumentCommand{\tomorrowof}{ m m m }
{
	\group_begin:
	\juliandate_reverse:x { \juliandate_calc:nnnn { #1 + 1 } { #2 } { #3 } { \use:n } }
	\day = \l_juliandate_day_int
	\month = \l_juliandate_month_int
	\year = \l_juliandate_year_int
	\today
	\group_end:
}
\ExplSyntaxOff


\usepackage{xcolor}
\usepackage{textcomp}

% listing
%\usepackage[utf8]{inputenc}
 
\usepackage{listings}
\usepackage{verbatim}
\usepackage{color}
 
\definecolor{codegreen}{rgb}{0,0.6,0}
\definecolor{codegray}{rgb}{0.5,0.5,0.5}
\definecolor{codepurple}{rgb}{0.58,0,0.82}
\definecolor{backcolour}{rgb}{0.95,0.95,0.92}
 
\lstdefinestyle{mystyle}{
    backgroundcolor=\color{backcolour},   
    commentstyle=\color{codegreen},
    keywordstyle=\color{magenta},
    numberstyle=\tiny\color{codegray},
    stringstyle=\color{codepurple},
    basicstyle=\footnotesize,
    breakatwhitespace=false,         
    breaklines=true,                 
    captionpos=b,                    
    keepspaces=true,                 
    numbers=left,                    
    numbersep=5pt,                  
    showspaces=false,                
    showstringspaces=false,
    showtabs=false,                  
    tabsize=4
}
 
\lstset{style=mystyle}
\lstset{extendedchars=\true}
\lstset{inputencoding=utf8}
\usepackage{adjustbox}
\begin{document} % конец преамбулы, начало документа
\begin{titlepage}
	\begin{center}
		ФЕДЕРАЛЬНОЕ  ГОСУДАРСТВЕННОЕ АВТОНОМНОЕ \\
		ОБРАЗОВАТЕЛЬНОЕ УЧРЕЖДЕНИЕ ВЫСШЕГО ОБРАЗОВАНИЯ\\
		«НАЦИОНАЛЬНЫЙ ИССЛЕДОВАТЕЛЬСКИЙ УНИВЕРСИТЕТ\\
		«ВЫСШАЯ ШКОЛА ЭКОНОМИКИ»
	\end{center}
	
	\begin{center}
		\textbf{Московский институт электроники и математики}
		
		\textbf{Им. А.Н.Тихонова НИУ ВШЭ}
	\end{center}
	\vspace{1ex}	
	\begin{center}
		Подчезерцев Алексей Евгеньевич, группа БИВ172
		
		Солодянкин Андрей Александрович, группа БИВ172
	\end{center}	
	\vspace{1ex}
	\begin{center}
		\textbf{Домашнее задание по базам данных}
	\end{center}	
	\vspace{2ex}
	\begin{center}
		студентов образовательной программы бакалавриата \\
		<<Информатика и вычислительная техника>> \\
		
	\end{center}
	\vspace{2ex}
	\begin{flushright}
		Выполнили: 
		
		\vspace{1ex}
		А.Е. Подчезерцев 
		
		\vspace{1ex}
		А.А. Солодянкин 
	\end{flushright}

	\vfill
	\begin{center}
		Москва \the\year \, г.
	\end{center}
\end{titlepage}
\tableofcontents
\pagebreak

\section{Анализ предметной области}

База данных создаётся для информационного обслуживания системы онлайн курсов.
БД должна содержать данные о пользователях, курсах, материалов курсов (лекции и тесты), а так же результаты прохождения курсов.
В соответствии с предметной областью система строится с учётом следующих особенностей:

\begin{itemize}
	\item Каждый пользователь может пройти любой курс и каждый курс может пройти много пользователей;
	\item В каждом курсе может быть много блоков, блок может быть только в одном курсе;
	\item Каждый блок состоит из лекционных материалов и проверочных работ;
	\item Преподаватель может создавать лекционные материалы, а ученик может их прочитать;
	\item Преподаватель может создавать и проверять проверочные работы, ученик может несколько раз решать проверочные задания, а ассистент может проверять работы учеников;
	\item Администратор системы может создавать курсы и назначать преподавателей;
	\item Преподаватель может назначать ассистентов на курс;
	\item На каждом курсе может быть неограниченное число преподавателей, ассистентов и учеников, при этом каждый пользователь может участвовать только с одной ролью.	
\end{itemize}

Сущности предметной области:

\begin{enumerate}
	\item \textbf{Пользователь}. Атрибуты: ФИО, логин, пароль, пол, почта;
	\item \textbf{Курс}. Атрибуты: названия, категория, видимость;
	\item \textbf{Блок}. Атрибуты: тема, видимость;
	\item \textbf{Лекция}. Атрибуты: название, содержание, продолжительность, видимость;
	\item \textbf{Проверочный материал}. Атрибуты: название, максимальный балл, продолжительность, видимость;
	\item \textbf{Попытка}. Атрибуты: содержание, оценка.
\end{enumerate}

Исходя из выявленных сущностей, построим ER–диаграмму (Рис. \ref{img:db_ER}). 

\begin{figure}[H]
	\centering		
	\includegraphics[width=\linewidth]{schemas/ER}
	\caption{ER диаграмма}\label{img:db_ER}
\end{figure}

\section{Анализ информационных задач и круга пользователей системы}

Определим группы пользователей, их основные задачи и запросы к БД:

\begin{enumerate}
	\item Администратор ресурса
	\begin{itemize}
		\item Регистрация новых пользователей;
		\item Создание новых курсов;
		\item Привязка преподавателей к курсу;
		\item Управление платформой и курсами.
	\end{itemize}

	\item Преподаватель курса
	\begin{itemize}
		\item Регистрация и приглашение новых учеников и ассистентов на курс;
		\item Создание и редактирование информации о курсе (блоки, лекции, проверочные работы);
		\item Просмотр прогресса учеников по курсу;
		\item Проверка работ учащихся курса.
	\end{itemize}

	\item Ассистент курса
	\begin{itemize}
		\item Проверка работ учеников на подконтрольных курсов;
		\item Просмотр результатов собственной проверки.
	\end{itemize}

	\item Ученик
	\begin{itemize}
		\item Запись на существующие курсы;
		\item Чтение и фиксация прогресса по лекциям;
		\item Доступ к материалам проверочных работ;
		\item Многократное выполнение и отправка попыток на проверку;
		\item Доступ к результатам проверки.
	\end{itemize}
\end{enumerate}

\section{Определение требований к операционной обстановке}
На основе результатов анализа предметной области можно приблизительно оценить объём памяти, требуемой для хранения данных.
Примем ориентировочно, что:
\begin{itemize}
	\item В системе зарегистрировано 1000 пользователей (по 0.25К на запись);
	\item В системе создано 10 курсов, каждый из которых состоит в среднем из 32 блоков (0.2К на информацию о курсе);
	\item Каждый блок состоит из 5 лекционных материалов и 2 проверочных работ (0.1К на информацию о блоке);
	\item Каждая лекция содержит 5К текстовой информации;
	\item Каждый проверочный материал занимает 1К;
	\item Ученик записан в среднем на 2 курса;
	\item Ученик читает все лекции и решает в среднем 2 раза каждое задание (0.1К на чтение лекции и 3К на попытку решения).
\end{itemize}

Тогда объем памяти, занимаемый базой данных будет примерно равен:
\begin{multline*}
2 \times( 10 \times ((32 \times (5 \times 5K + 2 \times 1K) + 0.1K) + 0.2K) + \\ 
+ 1000 \times (2 \times 32 \times (5 \times 0.1K + 2 \times 2 \times 3K) + 0.25K)) = 1617786K	\approx 1.5G
\end{multline*}

\section{Выбор СУБД и других программных средств}
Для реализации проекта БД была выбрана СУБД PostgreSQL.
Она имеет все необходимые и не только возможности для реализации данной работы (таблицы, представления, индексы, группы пользователей, соответствие стандартам ANSI SQL-92, SQL-99, SQL:2003, SQL:2011).
В отличии от MySQL, PostgreSQL имеет боле логичное поведение и предостерегает пользователя от возможных ошибок на более раннем этапе.
По сравнению с Oracle, PostgreSQL содержит понятные и человекочитаемые сообщения об ошибках, более приятную документацию, элегантные конструкции (например, для создания первичного ключа).
Кроме того, PostgreSQL занимает не так много места и не предоставляет дополнительных страданий при загрузке и установке на Windows/Linux.
Поэтому, будем писать на PostgreSQL.

\section{Преобразование ER–диаграммы в схему базы данных}

\begin{figure}[H]
	\centering		
	\includegraphics[width=\linewidth]{schemas/RDB}
	\caption{Схема базы данных}\label{img:RDB}
\end{figure}

\subsection{Составление реляционных отношений}

%TAGS: U, C, B, L, T, A, P, LR

%1. Пользователь. Атрибуты: ФИО, логин, пароль, пол, почта;

Потенциальными ключами отношения ПОЛЬЗОВАТЕЛЬ являются поля логин и адрес электронной почты. Все они занимают достаточно много места. Введём суррогатный первичный ключ Номер пользователя.

\begin{table}[H]
	\begin{flushleft} 
		\tablecaption{\label{tab:ClientV1} Схема отношения ПОЛЬЗОВАТЕЛЬ (Client) }
	\end{flushleft}
\begin{adjustbox}{width=\linewidth}
\begin{tabular}{|l|l|c|l|}
	\hline
	Содержание поля        & Имя поля  & Тип, длина & Примечания                          \\ \hline
	Номер пользователя     & U\_ID     &    N(4)    & \textbf{суррогатный первичный ключ} \\ \hline
	Фамилия, имя, отчество & U\_FIO    &   V(100)   & обязательное поле                   \\ \hline
	Дата рождения          & U\_BORN   &    Date    & обязательное поле                   \\ \hline
	Пол                    & U\_GENDER &    C(1)    & обязательное поле, 'м' или 'ж'      \\ \hline
	Логин                  & U\_LOGIN  &   V(30)    & обязательное поле, уникальное поле  \\ \hline
	Почта                  & U\_MAIL   &   V(30)    & обязательное поле, уникальное поле  \\ \hline
	Пароль                 & U\_PASS   &   V(30)    & обязательное поле                   \\ \hline
\end{tabular}
\end{adjustbox}
\end{table}

%2. Курс. Атрибуты: названия, категория, видимость;

Потенциальным ключом отношения КУРС является поле название. Оно занимает достаточно много места. Введём суррогатный первичный ключ номер курса.

\begin{table}[H]
	\begin{flushleft} 
		\tablecaption{\label{tab:CourseV1} Схема отношения КУРС (Course) }
	\end{flushleft}
\begin{adjustbox}{width=\linewidth}
	\begin{tabular}{|l|l|c|l|}
		\hline
		Содержание поля & Имя поля     & Тип, длина & Примечания                          \\ \hline
		Номер курса     & C\_ID         &    N(4)    & \textbf{суррогатный первичный ключ} \\ \hline
		Название        & C\_NAME       &   V(100)   & обязательное поле                   \\ \hline
		Категории       & C\_CATEGORY   &   V(100)   & обязательное поле                   \\ \hline
		Видимость       & C\_VISIBILITY &    BOOL    & обязательное поле, default FALSE    \\ \hline
	\end{tabular}
\end{adjustbox}
\end{table}

%3. Блок. Атрибуты: тема, видимость;

У отношения БЛОК нет потенциального ключа, поэтому введём суррогатный первичный ключ номер блока.

\begin{table}[H]
	\begin{flushleft} 
		\tablecaption{\label{tab:BlockV1} Схема отношения БЛОК (Block) }
	\end{flushleft}
\begin{adjustbox}{width=\linewidth}
	\begin{tabular}{|l|l|c|l|}
		\hline
		Содержание поля & Имя поля      & Тип, длина & Примечания                          \\ \hline
		Номер блока     & B\_ID         &    N(4)    & \textbf{суррогатный первичный ключ} \\ \hline
		Тема            & B\_THEME      &   V(100)   & обязательное поле                   \\ \hline
		Видимость       & B\_VISIBILITY &    BOOL    & обязательное поле, default FALSE    \\ \hline
		Курс            & B\_COURSE     &    N(4)    & внешний ключ (к Course)             \\ \hline
	\end{tabular}
\end{adjustbox}
\end{table}

%4. Лекция. Атрибуты: название, содержание, продолжительность, видимость;

У отношения ЛЕКЦИЯ нет потенциального ключа, поэтому введём суррогатный первичный ключ номер лекции.

\begin{table}[H]
	\begin{flushleft} 
		\tablecaption{\label{tab:LectureV1} Схема отношения ЛЕКЦИЯ (Lecture) }
	\end{flushleft}
\begin{adjustbox}{width=\linewidth}
	\begin{tabular}{|l|l|c|l|}
		\hline
		Содержание поля   & Имя поля      & Тип, длина & Примечания                          \\ \hline
		Номер лекции      & L\_ID         &    N(4)    & \textbf{суррогатный первичный ключ} \\ \hline
		Название          & L\_NAME       &   V(100)   & обязательное поле                   \\ \hline
		Содержание        & L\_CONTENT    &    TEXT    &                                     \\ \hline
		Продолжительность & L\_DURATION   &  INTERVAL  &                                     \\ \hline
		Видимость         & L\_VISIBILITY &    BOOL    & обязательное поле, default FALSE    \\ \hline
		Блок              & L\_BLOCK      &    N(4)    & внешний ключ (к Block)              \\ \hline
	\end{tabular}
\end{adjustbox}
\end{table}
%5. Проверочный материал. Атрибуты: название, максимальный балл, продолжительность, видимость;

У отношения ПРОВЕРОЧНЫЙ МАТЕРИАЛ нет потенциального ключа, поэтому введём суррогатный первичный ключ номер проверочного материала.

\begin{table}[H]
	\begin{flushleft} 
		\tablecaption{\label{tab:TestMaterialV1} Схема отношения ПРОВЕРОЧНЫЙ МАТЕРИАЛ (TestMaterial) }
	\end{flushleft}
\begin{adjustbox}{width=\linewidth}
	\begin{tabular}{|l|l|c|l|}
		\hline
		Содержание поля              & Имя поля     & Тип, длина & Примечания                          \\ \hline
		Номер проверочного материала & T\_ID         &    N(4)    & \textbf{суррогатный первичный ключ} \\ \hline
		Название                     & T\_NAME       &   V(100)   & обязательное поле                   \\ \hline
		Задание                      & T\_TASK       &    TEXT    & обязательное поле                   \\ \hline
		Максимальный балл            & T\_MAX        &    N(4)    & обязательное поле                   \\ \hline
		Продолжительность            & T\_DURATION   &  INTERVAL  &                                     \\ \hline
		Видимость                    & T\_VISIBILITY &    BOOL    & обязательное поле, default FALSE    \\ \hline
		Блок                         & T\_BLOCK      &    N(4)    & внешний ключ (к Block)              \\ \hline
	\end{tabular}
\end{adjustbox}
\end{table}

%6. Решение. Атрибуты: содержание, оценка.

У отношения ПОПЫТКА нет потенциального ключа, поэтому введём суррогатный первичный ключ номер решения.

\begin{table}[H]
	\begin{flushleft} 
		\tablecaption{\label{tab:AttemptV1} Схема отношения ПОПЫТКА (Attempt) }
	\end{flushleft}
\begin{adjustbox}{width=\linewidth}
	\begin{tabular}{|l|l|c|l|}
		\hline
		Содержание поля & Имя поля   & Тип, длина & Примечания                          \\ \hline
		Номер решения   & A\_ID      &    N(4)    & \textbf{суррогатный первичный ключ} \\ \hline
		Содержание      & A\_CONTENT &    TEXT    & обязательное поле                   \\ \hline
		Оценка          & A\_MARK    &    N(4)    &                                     \\ \hline
		Дата сдачи      & A\_DATE    &    Date    & обязательное поле, default NOW()    \\ \hline
		Автор попытки   & A\_AUTHOR  &    N(4)    & внешний ключ (к Client)               \\ \hline
		Проверяющий     & A\_CHECKER &    N(4)    & внешний ключ (к Client)               \\ \hline
		Номер задания   & A\_TASK    &    N(4)    & внешний ключ (к TestMaterial)       \\ \hline
	\end{tabular}
\end{adjustbox}
\end{table}

%7. Участие. Атрибуты: роль, номер пользователь, номер курса.

Вспомогательное отношение УЧАСТИЕ, первичным ключом является комбинация двух полей этого отношения. Можно вообще не вводить первичный ключ для данного отношения, т.к. на него никто не ссылается. Но уникальность этой комбинации является в данном случае ограничением целостности данных, поэтому мы возьмём эту комбинацию в качестве первичного ключа соответствующего отношения.

\begin{table}[H]
	\begin{flushleft} 
		\tablecaption{\label{tab:ParticipationV1} Схема отношения УЧАСТИЕ (Participation) }
	\end{flushleft}
\begin{adjustbox}{width=\linewidth}
	\begin{tabular}{|l|l|c|l|l|}
		\hline
		Содержание поля & Имя поля & Тип, длина & \multicolumn{2}{l|}{Примечания} \\ \hline
		Роль & P\_ROLE & С(10) & \multicolumn{2}{l|}{обязательное поле} \\ \hline
		Номер пользователя & P\_CLIENT & N(4) & \begin{tabular}[c]{@{}l@{}}
			  внешний ключ   \\
			(к Client)
		\end{tabular} & \multirow{2}{*}{\begin{tabular}[c]{@{}l@{}}\textbf{составной}\\ \textbf{первичный} \\ \textbf{ключ}\end{tabular}} \\ \cline{1-4}
		Номер курса & P\_COURSE & N(4) & \begin{tabular}[c]{@{}l@{}}внешний ключ \\ (к Course)\end{tabular} &  \\ \hline
	\end{tabular}
\end{adjustbox}
\end{table}

%8.  ПРОЧИТАННЫЕ ЛЕКЦИИ. Атрибуты: номер пользователя, номер лекции.

Вспомогательное отношение  ПРОЧИТАННЫЕ ЛЕКЦИИ, первичным ключом является комбинация двух полей этого отношения. Можно вообще не вводить первичный ключ для данного отношения, т.к. на него никто не ссылается. Но уникальность этой комбинации является в данном случае ограничением целостности данных, поэтому мы возьмём эту комбинацию в качестве первичного ключа соответствующего отношения.

\begin{table}[H]
	
	\begin{flushleft} 
		\tablecaption{\label{tab:LectionReadV1} Схема отношения ПРОЧИТАННЫЕ ЛЕКЦИИ (LectionRead) }
	\end{flushleft}
	\begin{adjustbox}{width=\linewidth}
	\begin{tabular}{|l|l|c|l|l|}
		\hline
		Содержание поля & Имя поля & Тип, длина & \multicolumn{2}{l|}{Примечания} \\ \hline
		Номер пользователя & LR\_CLIENT & N(4) & \begin{tabular}[c]{@{}l@{}}внешний ключ \\ (к Client)\end{tabular} & \multirow{2}{*}{\begin{tabular}[c]{@{}l@{}}\textbf{составной}\\ \textbf{первичный} \\ \textbf{ключ}\end{tabular}} \\ \cline{1-4}
		Номер лекции & LR\_NUMLECTURE & N(4) & \begin{tabular}[c]{@{}l@{}}
			  внешний ключ    \\
			(к номер Lecture)
		\end{tabular} &  \\ \hline
	\end{tabular}
\end{adjustbox}
\end{table}


\subsection{Нормализация полученных отношений (до 4НФ)}

\textbf{1НФ}
Для приведения к первой нормальной форме значение многозначного поля «Категории» в отношении КУРС было вынесено в отдельное справочное отношение. Помимо этого поле «ФИО» в отношении ПОЛЬЗОВАТЕЛЬ было разделено на 3 поля: обязательное «Фамилия», обязательное «Имя» и необязательное «Отчество».

\textbf{2НФ}
Базу данных не требуется приводить ко второй нормальной форме, так как все отношения, имеющие составные первичные ключи, удовлетворяют условию второй нормальной формы.

\textbf{3НФ}
Для приведение к 3НФ необходимо выявить транзитивные зависимости. Транзитивная зависимость выявлена для «Продолжительность» из отношения ЛЕКЦИЯ. Продолжительность лекции зависит от объёма информации в поле «Содержание», значит можно убрать поле «Продолжительность» и создать представление, отражающие время изучения лекции.

\textbf{4НФ}
Приведение к 4НФ не требуется, так как во всех отношениях отсутствуют нетривиальные многозначные зависимости, значит все отношения удовлетворяют условию 4НФ.


\subsection{Полученные после нормализации отношения}


%TAGS: U, C, B, L, T, A, P, LR, CT

%1. Пользователь

\begin{table}[H]
	\begin{flushleft} 
		\tablecaption{\label{tab:ClientV2} Схема отношения ПОЛЬЗОВАТЕЛЬ (Client) }
	\end{flushleft}
	\begin{adjustbox}{width=\linewidth}
		\begin{tabular}{|l|l|c|l|}
			\hline
			Содержание поля    & Имя поля  & Тип, длина & Примечания                          \\ \hline
			Номер пользователя & U\_ID     &    N(4)    & \textbf{суррогатный первичный ключ} \\ \hline
			Фамилия            & U\_LNAME  &   V(30)    & обязательное поле                   \\ \hline
			Имя                & U\_FNAME  &   V(30)    & обязательное поле                   \\ \hline
			Отчество           & U\_PATRO  &   V(30)    &                                     \\ \hline
			Дата рождения      & U\_BORN   &    Date    & обязательное поле                   \\ \hline
			Пол                & U\_GENDER &    C(1)    & обязательное поле, 'м' или 'ж'      \\ \hline
			Логин              & U\_LOGIN  &   V(30)    & обязательное поле, уникальное поле  \\ \hline
			Почта              & U\_MAIL   &   V(30)    & обязательное поле, уникальное поле  \\ \hline
			Пароль             & U\_PASS   &   V(30)    & обязательное поле                   \\ \hline
		\end{tabular}
	\end{adjustbox}
\end{table}

%2. КУРС

\begin{table}[H]
	\begin{flushleft} 
		\tablecaption{\label{tab:CourseV2} Схема отношения КУРС (Course) }
	\end{flushleft}
\begin{adjustbox}{width=\linewidth}
	\begin{tabular}{|l|l|c|l|}
		\hline
		Содержание поля & Имя поля     & Тип, длина & Примечания                          \\ \hline
		Номер курса     & C\_ID         &    N(4)    & \textbf{суррогатный первичный ключ} \\ \hline
		Название        & C\_NAME       &   V(100)   & обязательное поле                   \\ \hline
		%Категории       & C\_CATEGORY   &   V(100)   & обязательное поле                   \\ \hline
		Видимость       & C\_VISIBILITY &    BOOL    & обязательное поле, default FALSE    \\ \hline
	\end{tabular}
\end{adjustbox}
\end{table}

%3. КАТЕГОРИЯ. справочная таблица

У справочного отношения КАТЕГОРИЯ есть потенциальный первичный ключ «Название категории», но название категории может меняться, поэтому введём суррогатный первичный ключ номер категории.

\begin{table}[H]
	\begin{flushleft} 
		\tablecaption{\label{tab:CategoryV2} Схема отношения КАТЕГОРИЯ (Category) }
	\end{flushleft}
\begin{adjustbox}{width=\linewidth}
	\begin{tabular}{|l|l|c|l|}
		\hline
		Содержание поля    & Имя поля & Тип, длина & Примечания                          \\ \hline
		Номер категории    & CT\_ID   &    N(4)    & \textbf{суррогатный первичный ключ} \\ \hline
		Название категории & CT\_NAME &   V(30)    & обязательное поле                   \\ \hline
	\end{tabular}
\end{adjustbox}
\end{table}

%4. КАТЕГОРИЯ КУРСА, номер курса, номер категории

Вспомогательное отношение  КАТЕГОРИЯ КУРСА, первичным ключом является комбинация двух полей этого отношения. Можно вообще не вводить первичный ключ для данного отношения, т.к. на него никто не ссылается. Но уникальность этой комбинации является в данном случае ограничением целостности данных, поэтому мы возьмём эту комбинацию в качестве первичного ключа соответствующего отношения.

\begin{table}[H]
	
	\begin{flushleft} 
		\tablecaption{\label{tab:CategoryCourseV2} Схема отношения КАТЕГОРИЯ КУРСА (CategoryCourse) }
	\end{flushleft}
	\begin{adjustbox}{width=\linewidth}
		\begin{tabular}{|l|l|c|l|l|}
			\hline
			Содержание поля & Имя поля & Тип, длина & \multicolumn{2}{l|}{Примечания} \\ \hline
			Номер курса & CC\_COURSE & N(4) & \begin{tabular}[c]{@{}l@{}}внешний ключ \\ (к Course)\end{tabular} & \multirow{2}{*}{\begin{tabular}[c]{@{}l@{}}\textbf{составной}\\ \textbf{первичный} \\ \textbf{ключ}\end{tabular}} \\ \cline{1-4}
			Номер категории & CC\_CATEGORY & N(4) & \begin{tabular}[c]{@{}l@{}}
				  внешний ключ    \\
				(к номер Category)
			\end{tabular} &  \\ \hline
		\end{tabular}
	\end{adjustbox}
\end{table}

%5 Блок. Атрибуты: тема, видимость;

\begin{table}[H]
	\begin{flushleft} 
		\tablecaption{\label{tab:BlockV2} Схема отношения БЛОК (Block) }
	\end{flushleft}
	\begin{adjustbox}{width=\linewidth}
		\begin{tabular}{|l|l|c|l|}
			\hline
			Содержание поля & Имя поля & Тип, длина & Примечания                          \\ \hline
			Номер блока     &  B\_ID        &    N(4)    & \textbf{суррогатный первичный ключ} \\ \hline
			Тема            &  B\_THEME        &   V(100)   & обязательное поле                   \\ \hline
			Видимость       &  B\_VISIBILITY        &    BOOL    & обязательное поле, default FALSE    \\ \hline
			Курс            &  B\_COURSE        &    N(4)    & внешний ключ (к Course)             \\ \hline
		\end{tabular}
	\end{adjustbox}
\end{table}

%6. Лекция. Атрибуты: название, содержание, продолжительность, видимость;

\begin{table}[H]
	\begin{flushleft} 
		\tablecaption{\label{tab:LectureV2} Схема отношения ЛЕКЦИЯ (Lecture) }
	\end{flushleft}
	\begin{adjustbox}{width=\linewidth}
		\begin{tabular}{|l|l|c|l|}
			\hline
			Содержание поля   & Имя поля      & Тип, длина & Примечания                          \\ \hline
			Номер лекции      & L\_ID         &    N(4)    & \textbf{суррогатный первичный ключ} \\ \hline
			Название          & L\_NAME       &   V(100)   & обязательное поле                   \\ \hline
			Содержание        & L\_CONTENT    &    TEXT    &                                     \\ \hline
			%Продолжительность & L\_DURATION   &  INTERVAL  &                                     \\ \hline
			Видимость         & L\_VISIBILITY &    BOOL    & обязательное поле, default FALSE    \\ \hline
			Блок              & L\_BLOCK      &    N(4)    & внешний ключ (к Block)              \\ \hline
		\end{tabular}
	\end{adjustbox}
\end{table}

%7. Проверочный материал. Атрибуты: название, максимальный балл, продолжительность, видимость;

\begin{table}[H]
	\begin{flushleft} 
		\tablecaption{\label{tab:TestMaterialV2} Схема отношения ПРОВЕРОЧНЫЙ МАТЕРИАЛ (TestMaterial) }
	\end{flushleft}
	\begin{adjustbox}{width=\linewidth}
		\begin{tabular}{|l|l|c|l|}
			\hline
			Содержание поля              & Имя поля     & Тип, длина & Примечания                          \\ \hline
			Номер проверочного материала & T\_ID         &    N(4)    & \textbf{суррогатный первичный ключ} \\ \hline
			Название                     & T\_NAME       &   V(100)   & обязательное поле                   \\ \hline
			Задание                      & T\_TASK       &    TEXT    & обязательное поле                   \\ \hline
			Максимальный балл            & T\_MAX        &    N(4)    & обязательное поле                   \\ \hline
			Продолжительность            & T\_DURATION   &  INTERVAL  &                                     \\ \hline
			Видимость                    & T\_VISIBILITY &    BOOL    & обязательное поле, default FALSE    \\ \hline
			Блок                         & T\_BLOCK      &    N(4)    & внешний ключ (к Block)              \\ \hline
		\end{tabular}
	\end{adjustbox}
\end{table}

%8. Решение. Атрибуты: содержание, оценка.

\begin{table}[H]
	\begin{flushleft} 
		\tablecaption{\label{tab:AttemptV2} Схема отношения ПОПЫТКА (Attempt) }
	\end{flushleft}
	\begin{adjustbox}{width=\linewidth}
		\begin{tabular}{|l|l|c|l|}
			\hline
			Содержание поля & Имя поля   & Тип, длина & Примечания                          \\ \hline
			Номер решения   & A\_ID      &    N(4)    & \textbf{суррогатный первичный ключ} \\ \hline
			Содержание      & A\_CONTENT &    TEXT    & обязательное поле                   \\ \hline
			Оценка          & A\_MARK    &    N(4)    &                                     \\ \hline
			Дата сдачи      & A\_DATE    &    Date    & обязательное поле, default NOW()    \\ \hline
			Автор попытки   & A\_AUTHOR  &    N(4)    & внешний ключ (к Client)               \\ \hline
			Проверяющий     & A\_CHECKER &    N(4)    & внешний ключ (к Client)               \\ \hline
			Номер задания   & A\_TASK    &    N(4)    & внешний ключ (к TestMaterial)       \\ \hline
		\end{tabular}
	\end{adjustbox}
\end{table}

%9. Участие. Атрибуты: роль, номер пользователь, номер курса.

\begin{table}[H]
	\begin{flushleft} 
		\tablecaption{\label{tab:ParticipationV2} Схема отношения УЧАСТИЕ (Participation) }
	\end{flushleft}
	\begin{adjustbox}{width=\linewidth}
		\begin{tabular}{|l|l|c|l|l|}
			\hline
			Содержание поля & Имя поля & Тип, длина & \multicolumn{2}{l|}{Примечания} \\ \hline
			Роль & P\_ROLE & С(10) & \multicolumn{2}{l|}{обязательное поле} \\ \hline
			Номер пользователя & P\_CLIENT & N(4) & \begin{tabular}[c]{@{}l@{}}
				внешний ключ   \\
				(к Client)
			\end{tabular} & \multirow{2}{*}{\begin{tabular}[c]{@{}l@{}}\textbf{составной}\\ \textbf{первичный} \\ \textbf{ключ}\end{tabular}} \\ \cline{1-4}
			Номер курса & P\_COURSE & N(4) & \begin{tabular}[c]{@{}l@{}}внешний ключ \\ (к Course)\end{tabular} &  \\ \hline
		\end{tabular}
	\end{adjustbox}
\end{table}

%10.  ПРОЧИТАННЫЕ ЛЕКЦИИ. Атрибуты: номер пользователя, номер лекции.

\begin{table}[H]
	\begin{flushleft} 
		\tablecaption{\label{tab:LectionReadV2} Схема отношения ПРОЧИТАННЫЕ ЛЕКЦИИ (LectionRead) }
	\end{flushleft}
	\begin{adjustbox}{width=\linewidth}
		\begin{tabular}{|l|l|c|l|l|}
			\hline
			Содержание поля & Имя поля & Тип, длина & \multicolumn{2}{l|}{Примечания} \\ \hline
			Номер пользователя & LR\_CLIENT & N(4) & \begin{tabular}[c]{@{}l@{}}внешний ключ \\ (к Client)\end{tabular} & \multirow{2}{*}{\begin{tabular}[c]{@{}l@{}}\textbf{составной}\\ \textbf{первичный} \\ \textbf{ключ}\end{tabular}} \\ \cline{1-4}
			Номер лекции & LR\_NUMLECTURE & N(4) & \begin{tabular}[c]{@{}l@{}}
				внешний ключ    \\
				(к номер Lecture)
			\end{tabular} &  \\ \hline
		\end{tabular}
	\end{adjustbox}
\end{table}

\begin{figure}[H]
	\centering		
	\includegraphics[width=\linewidth]{schemas/RDB_normalized}
	\caption{Окончательная схема БД платформы онлайн курсов}\label{img:RDB_normalized}
\end{figure}

\section{Определение дополнительных ограничений целостности}
Перечислим ограничения целостности, которые не были ранее указаны.
\begin{enumerate}
	\item Дата рождения пользователя должная быть менее текущей даты;
	\item Адрес почты должен удовлетворять регулярному выражению 
	
	\begin{lstlisting}	
	^[A-Za-z0-9._-]+@[A-Za-z0-9.-]+$
	\end{lstlisting}
	
	Заметим, что после после проверки могут быть сохранены неправильные варианты, однако данный фильтр пропустит все верные значения.
	Проверять почтовый адрес с помощью регулярных выражений не является лучшей практикой, более элегантное решение -- проверить адрес путем отправки письма на него;
	\item  Продолжительность лекции и проверочного материала могут быть только положительными интервалами;
	\item Максимальный балл проверочного материала есть неотрицательное число;
	\item Оценка за попытку решения должна быть неотрицательным числом, не превышающим максимальный балл за соответствующее задание;
	\item Дата сдачи всегда автоматически выставляется в NOW();
	\item Атрибут Роль в таблице участие может принимать одно из 3 значений: <<Teacher>>, <<Assistant>>, <<Student>>;
	\item Пользователь может помечать прочитанными только те лекции, которые принадлежат курсу, в который пользователь записан с правами Student;
	\item Пользователь может решать только те задания, которые принадлежат курсу, в который пользователь записан с правами Student;
	\item Пользователь может проверять только те попытки, которые принадлежат курсу, в который пользователь записан с правами <<Teacher>> или <<Assistant>>.
\end{enumerate}

Последние три ограничения разрывают два цикла в схеме базы данных.
Их нельзя реализовать в схеме отношение и они будут реализованы с помощью приложения или триггера.

\section{Описание групп пользователей и прав доступа}

На первый взгляд, пользователи получат мало прав  к таблицам, однако это сделано с целью повышения безопасности и сохранения пользовательских данных.
Доступ к информации будет организован через представления.

\begin{table}[H]
	\begin{flushleft} 
		\tablecaption{\label{tab:roles} Права доступа}
	\end{flushleft}
\begin{tabular}{|c|c|c|c|c|}
	\hline
	                     &   \multicolumn{4}{c|}{Группы пользователей (роли)}   \\ \cline{2-5}
	      Таблицы        & Администратор & Преподаватели & Ассистенты & Ученики \\
	                     &    ресурса    &     курса     &   курса    &         \\ \hline
	    Пользователь     &     SIUD      &       I       &            &         \\ \hline
	        Курс         &     SIUD      &      SU       &            &         \\ \hline
	     Категория       &     SIUD      &       S       &     S      &    S    \\ \hline
	  Категория курса    &       S       &     SIUD      &     S      &    S    \\ \hline
	        Блок         &       S       &     SIUD      &            &         \\ \hline
	       Лекция        &       S       &     SIUD      &            &         \\ \hline
	Проверочный материал &       S       &     SIUD      &            &         \\ \hline
	      Попытка        &       S       &      SUD      &     SU     &         \\ \hline
	      Участие        &     SIUD      &     SIUD      &            &         \\ \hline
	 Прочитанные лекции  &       S       &     SIUD      &            &   SI    \\ \hline
\end{tabular}
\end{table}

\section{Реализация проекта базы данных}

\subsection{Создание таблиц}

{\small \verbatiminput{code/tab_client.sql}}
{\small \verbatiminput{code/tab_course.sql}}
{\small \verbatiminput{code/tab_category.sql}}
{\small \verbatiminput{code/tab_categoryCourse.sql}}
{\small \verbatiminput{code/tab_block.sql}}
{\small \verbatiminput{code/tab_lecture.sql}}
{\small \verbatiminput{code/tab_testMaterial.sql}}
{\small \verbatiminput{code/tab_Attempt.sql}}
{\small \verbatiminput{code/tab_participation.sql}}
{\small \verbatiminput{code/tab_lectureRead.sql}}

\subsection{Создание представлений}

Создадим функцию для получения идентификатора текущего пользователя.

{\small \verbatiminput{code/fun_userID.sql}}

Представления для безопасного получения информации пользователем:

{\small \verbatiminput{code/view_my.sql}}
Представления для безопасного получения информации о курсах и пользователях системы:

{\small \verbatiminput{code/view_public.sql}}

Представления для получение информации об участниках курсов с различными ролями:

{\small \verbatiminput{code/view_course.sql}}

\begin{table}[H]
	\begin{flushleft} 
		\tablecaption{\label{tab:role_view}Права доступа к представлениям}
	\end{flushleft}
	\begin{adjustbox}{width=\linewidth}
	\begin{tabular}{|c|c|c|c|c|}
		\hline
		& \multicolumn{4}{c|}{Группы пользователей (роли)}     \\ \cline{2-5} 
		Представление          & Администратор & Преподаватели & Ассистенты & Ученики \\
		& ресурса       & курса         & курса      &         \\ \hline
		Мои курсы              & S             & S             & S          & S       \\ \hline
		Мои блоки              & S             & S             & S          & S       \\ \hline
		Мои лекции             & S             & S             & S          & S       \\ \hline
		Мои тесты              & S             & S             & S          & S       \\ \hline
		Мои попытки            & S             & S             & S          & SIU     \\ \hline
		Мои прочитанные лекции & S             & S             & S          & S       \\ \hline
		Публичные курсы        & S             & S             & S          & S       \\ \hline
		Публичные пользователи & S             & S             & S          & S       \\ \hline
		Преподаватели курсов   & S             & S             & S          & S       \\ \hline
		Ассистенты курсов      & S             & S             & S          & S       \\ \hline
		Ученики курсов         & S             & S             & S          & S       \\ \hline
	\end{tabular}
	\end{adjustbox}
\end{table}

\subsection{Назначение прав доступа}

Создадим группы пользователей:

{\small \verbatiminput{code/roles.sql}}

Назначим права доступа к данным:

{\small \verbatiminput{code/grant.sql}}

\subsection{Создание индексов}

{\small \verbatiminput{code/index.sql}}
%\newpage 
%\renewcommand{\refname}{{\normalsize СПИСОК ИСПОЛЬЗОВАННЫХ ИСТОЧНИКОВ}} 
%\centering 
%\begin{thebibliography}{9} 
%	\addcontentsline{toc}{section}{\refname} 
%	\bibitem{sql} Beaulieu A. Learning SQL: Master SQL Fundamentals. – " O'Reilly Media, Inc.", 2009.
%	
%\end{thebibliography}

\end{document} % конец документа